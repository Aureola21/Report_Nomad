\documentclass{beamer}
\usetheme{Madrid}
\usepackage{graphicx}
\usepackage{amsmath}
\usepackage{hyperref}

\title[NoMaD]{NoMaD: \textbf{N}avigati\textbf{o}n with Goal-\textbf{Ma}sked \textbf{D}iffusion}
\author{Sehaj Ganjoo, Shobhnik Kriplani, \\ Abhishek Kumar Jha, Namashivayaa V}
\institute{IISc Bengaluru \newline BTech. Mathematics and Computing}
\date{April 2025}

\begin{document}

\begin{frame}
\titlepage
\end{frame}

\begin{frame}{Motivation and Goal}
Robotic learning for navigation in unfamiliar environments requires:\\
\begin{itemize}
    \item The ability to perform task-oriented navigation
    \item Task-agnostic exploration 
\end{itemize}
\begin{block}{Issue}
    Traditionally, these functionalities are tackled by separate systems.
    Exploration problem can be factorized into:\\
    \begin{itemize}
        \item \textbf{Local Exploration strategies} Objective: Learn control policies that can take diverse, short-horizon actions.
        \item \textbf{Global Exploration strategies} Objective: A high level planner that uses the policies for long-horizon goal-seeking. (Efficient Mapping?)
    \end{itemize}
\end{block}
\begin{block}{Solution}
    Maybe, use a single model to perform both tasks?
\end{block}
\end{frame}
\begin{frame}{What is NoMaD?}
    NoMaD is a transformer-based diffusion policy designed for long-horizon, memory-based
    navigation, that is capable of both \textbf{goal-conditioned navigation} and \textbf{open-ended exploration}.

    \texttt{NoMaD = EfficientNet + Vision Transformer + Diffusion Policies}
\begin{itemize}
    \item Can we improve trajectory prediction in robot navigation using denoising diffusion models?
    \item How can transformer-based memory and temporal context improve performance?
\end{itemize}
\end{frame}

\begin{frame}{Literature Survey}
\begin{itemize}
    \item \textbf{ViNT (Janner et al., 2022):} Vision Transformer for long-horizon visual navigation.
    \item \textbf{Diffusion Policies:} Denoising Diffusion Probabilistic Models (DDPM) for behavior cloning.
    \item \textbf{NOMAD:} Combines ViNT perception with diffusion-based trajectory decoding.
\end{itemize}
\end{frame}

\begin{frame}{Architecture Overview}
\begin{itemize}
    \item \textbf{Perception:} EfficientNet-B0 backbone $\rightarrow$ Transformer-based temporal encoder.
    \item \textbf{Diffusion Decoder:} Conditional UNet1D for generating waypoint sequences.
    \item \textbf{Action Decoder:} Maps waypoints to low-level actions.
\end{itemize}
% \includegraphics[width=0.8\linewidth]{architecture.png}
\end{frame}

\begin{frame}{Training Procedure}
\begin{itemize}
    \item Dataset: SACSoN / RECON / GoStanford
    \item Batch size: 32, Epochs: 100
    \item Optimizer: AdamW, LR: $10^{-4}$
    \item Scheduler: Cosine annealing
    \item Loss: MSE on predicted noise + temporal distance
\end{itemize}
\[ \mathcal{L}_{NoMaD} = MSE(\epsilon, \hat{\epsilon}) + \lambda \cdot MSE(d(o_t, o_g), f_d(c_t)) \]
\end{frame}

\begin{frame}{Experiments and Results}
\textbf{Metrics:}
\begin{itemize}
    \item Diffusion Loss $\approx 1.11$
    \item Distance Loss $\approx 128$
    \item Cosine Similarity $\approx 0.47$
\end{itemize}
\textbf{Comparison with ViNT:}
\begin{itemize}
    \item Similar performance in goal-conditioned tasks
    \item No performance degradation when adding diffusion
\end{itemize}
\end{frame}

\begin{frame}{Challenges Faced}
\begin{itemize}
    \item CUDA Out Of Memory errors on limited GPU
    \item Module import issues with nested folder structures
    \item Gradients not propagating due to detached variables
\end{itemize}
\end{frame}

\begin{frame}{Team Contributions}
\textbf{Sehaj Ganjoo:}
\begin{itemize}
    \item Set up training pipeline and environment
    \item Integrated and debugged diffusion model
    \item Wrote training script and logging tools
    \item Conducted experiments and generated plots
    \item Created report and presentation
\end{itemize}
\end{frame}

\begin{frame}{Conclusion and Future Work}
\begin{itemize}
    \item Successfully trained NOMAD using diffusion for visual navigation
    \item Showed compatibility with ViNT-based perception
    \item Future work:
    \begin{itemize}
        \item Evaluate in simulation / real-world
        \item Improve runtime performance
        \item Try larger ViTs and alternate decoders
    \end{itemize}
\end{itemize}
\end{frame}

\begin{frame}{Q\&A}
Thank you! \newline
\textit{Questions are welcome.}
\end{frame}

\end{document}